\section{Background}
\label{sec:Background}

\subsection{What is a 'message'?}
\label{sub:What is a 'message'?}

In todays interconnected world, computer programs rarely exist in a vacuum.
Rather, they form one small part of a much larger \gls{soa} -
consisting of multiple services, jobs and scripts, all exchanging information in
the form of messages. These messages may adhere to a standard data-interchange
format (for example, \gls{json}). They may correspond to an agreed upon
specification (for example, \href{https://goo.gl/rjuP4C}{IEEE 1671-2010}).
Or they may simply be blobs of binary information transmitted over the network -
completely subject to the interpretation of the sender and received.
At a fundamental level, though, a message is simply a collection of bytes, to be
transmitted from point A, to point B.

\subsection{What is a message broker?}
\label{sub:What is a message broker?}

\begin{figure}
  \input{figures/directMessaging}
  \caption{Two services, A and B, directly exchanging messages}
  \label{fig:tikz:directMessaging}
\end{figure}

To understand the role message brokers typically play in \glspl{soa}, we first
examine the simplest method of transmitting bytes between two applications -
directly transmitting messages between two applications (shown in
Figure~\ref{fig:tikz:directMessaging} ).

In this example, the application 'A' wishes to transmit a simple message
(a sequence of bytes) to application 'B', and does so in the simplest method
possible. This could involve making an \gls{rpc}, opening a Unix/\gls{tcp} socket,
or making a HTTP web request - for the purposes of this illustration the exact
mechanism by which bytes are transferred is unimportant, the fact
that the transfer takes place \emph{directly} between the two parties is all that
matters.

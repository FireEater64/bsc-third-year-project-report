\chapter{Conclusion}
\label{chap:Conclusion}

Overall, the project was an excellent learning experience. It provided an
opportunity to learn a new programming language and explore modern, open-source
development techniques - as well as learn about the intricacies involved with
building infrastructure software, the complexities of which are often
overlooked.

\section{Project Goals}
\label{sec:goalsConclusion}

The goal of this project has changed multiple times over the last 8 months.
Initial plans, were to build a very simple message broker - to be used as a
vehicle for exploring autonomic computing principals, and create a
'self-healing', 'self-improving' message broker. These goals were made with no
knowledge of the challenges associated with learning a new language, and in
dealing with the complexities of parallelism at the scale used in gamq. During a
mid-project review it was decided that, whilst it was theoretically possible
that the complexity of the broker could have been scaled down, and the original
project goals still met - the goal of the project should be refocused towards
producing a mature, high-performance, well-tested messaging broker. This
reflected not only the reality of the situation at the time, but a change in the
authors interests. These new goals, the ones outlined in
Chapter~\ref{chap:Introduction} have, as far as the author is concerned, all
been met - and the decision to change the project focus, was the correct one.

\subsection{Experience with GoLang}
\label{sub:golangConclusion}

After completing the authors first substantial piece of work with the Go
programming language, the power of CSP-style concurrency, combined with a clean,
modern development infrastructure - demonstrate why Go is now one of the top 20
most popular languages on GitHub\cite{languageRankings}, despite its young age.
Whilst it is arguable that a broker with similar feature-set could have been
produced in a more familiar language\footnote{Such as Python, or Java} in less
time, the performance and scalability of the implementation would have almost
certainly lagged behind that of gamq.

\subsection{Performance}
\label{sub:performance}

Whilst the performance of gamq (Section~\ref{sec:systemPerformance}) is
considerably higher than originally hoped (original estimates predicted
somewhere in the region of 5,000 messages/second), there are still numerous
potential performance increases that could be explored. The biggest of these
(according to the CPU performance profile in
Appendix~\ref{appendix:profileResults}) lies in reducing the amount of garbage
created by the application (and consequently, the amount of time spent
performing garbage collection), through the reuse of message objects, and the
use of zero-allocation buffers \cite{highPerformanceSystemsInGo} when receiving
data.

\section{Future Work}
\label{sec:Future Work}

Going forward, it is hoped that some of the knowledge (and possibly code!) gained
from this project continue to see use in some form or another. Towards the end
of the project, the 'professional open-source' message broker
NATS\footnote{\url{https://nats.io/}} was examined for similarities/differences
with gamq. Whilst the size and scope of NATS far outstrips that of gamq, it is
hoped that it will be possible to transfer some of the knowledge gained during
this project in the form of open-source
contributions\footnote{\url{https://github.com/nats-io/gnatsd/pulls}}
to other projects like NATS.

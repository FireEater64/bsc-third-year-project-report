\chapter{Design}
\label{chap:Design}

\section{Project philosophy}
\label{sec:Project philosophy}

Brief overview of the open development mentality ().

\section{Presentation interface}
\label{sec:presentation}

As a (primarily) middleware project - coming up with a compelling presentation
format is crucial. Broker metrics such as:

\begin{itemize}
  \item Messages/second (broken down by topic)
  \item (Average) End-to-end latency for each message
  \item Memory/Disk usage
  \item CPU utilization
  \item Pending messages
  \item Unacknowledged messages
  \item Total messages processed
\end{itemize}

And client metrics such as:

\begin{itemize}
  \item Messages sent
  \item Messages received
  \item Messages lost
\end{itemize}

Should be easily accessible, and visualized.

\section{Configuration}
\label{sec:Configuration}

As with most pieces of software - there are situations where the default
behavior of a message broker requires adjustment to suit its operating
environment. One example of this is the network port that the broker operates
on, and which other applications will connect to in order to exchange messages.
Whilst the default port (48879\footnote{A number chosen due to its memorable
hexadecimal representation: \texttt{0xBEEF}} in the case of this broker) may be
suitable in most cases, since no other 'well-known' applications use that
particular
port.\footnote{\url{https://en.wikipedia.org/wiki/List_of_TCP_and_UDP_port_numbers}}.

\subsection{Command-line configuration}
\label{sub:Command-line configuration}

A number of these configurable parameters make sense to expose as command-line
parameters, specified at application start time. The available parameters can be
exposed using the \texttt{--help} parameter, the output of which can be seen in
Listing~\ref{lst:gamqHelpOutput}

\begin{listing}[ht]
  \centering
  \inputminted{bash}{code/gamqHelpOutput}
  \caption{Output of running the broker with the --help flag}
  \label{lst:gamqHelpOutput}
\end{listing}

\subsection{File-based configuration}
\label{sub:File-based configuration}

Whilst specifying arguments on the command line gives flexibility, there are
certain options that, whilst configurable, are either too numerous, or change
too infrequently to justify command-line flags. One major example of this is the
log configuration for the broker - specifying the format, and location of
messages logged using the \href{https://github.com/cihub/seelog}{Seelog}
library. An example log configuration can be seen in
Listing~\ref{lst:seelogConfig}.

\begin{listing}[ht]
  \centering
  \inputminted{xml}{code/gamq/config/logconfig.xml}
  \caption{Example Seelog configuration file for gamq.}
  \label{lst:seelogConfig}
\end{listing}

\section{Code Structure}
\label{sec:codestructure}

\todo[inline]{Box-level overview of how the code is structured/how messages
flow through the broker}

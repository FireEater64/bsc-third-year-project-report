\chapter{Design}
\label{chap:Design}

\section{Project philosophy}
\label{sec:Project philosophy}

Brief overview of the open development mentality ().

\section{Presentation interface}
\label{sec:presentation}

As a (primarily) middleware project - coming up with a compelling presentation
format is crucial. Broker metrics such as:

\begin{itemize}
  \item Messages/second (broken down by topic)
  \item (Average) End-to-end latency for each message
  \item Memory/Disk usage
  \item CPU utilisation
  \item Pending messages
  \item Unacknowledged messages
  \item Total messages processed
\end{itemize}

And client metrics such as:

\begin{itemize}
  \item Messages sent
  \item Messages received
  \item Messages lost
\end{itemize}

Should be easily accessible, and visualised. The presentation and metrics
collection for gamq consists of the following components:

\subsection{StatsD}
\label{sub:StatsD}

StatsD is an open-source statistics daemon, written by developers at Etsy in
2010\cite{statsd}, and which has seen widespread adoption by infrastructure
teams ever since. StatsD aims to define a simple protocol
(Listing~\ref{lst:statsdSchema}) for sending statistics over the network, and
make it simple to aggregate and store those metrics for real-time analysis.
StatsD can differentiate between metrics of different types:

\begin{listing}
  \centering
  \mintinline{text}{metricname:1|c}
  \caption{A simple counter metric, exactly as it would appear inside a network packet}
  \label{lst:statsdSchema}
\end{listing}

\begin{description}
  \item[Counters] A simple counter that can be incremented or decremented. An
  example use case could be the number of messages sent through a broker.
  \item[Timing] The amount of time taken to complete a task. An example could be
  the amount of time taken to process a message.
  \item[Gauges] An arbitrary value, stored until it is superseded by a new
  value. An example of this could be the number of connections open to the
  broker at a single point in time.
\end{description}

These packets can be sent over the network as either TCP or UDP packets. Packets
can contain multiple metrics, separated with newline characters
(Listing~\ref{lst:statsdMultiplePackets}), which can greatly improve the
efficiency of the transfer, as the size of each metric (the order of a few
bytes) is far below the maximum capacity of a single packet.\footnote{The
typical \gls{mtu} of Ethernet networks is ~1500 bytes} Gamq uses the UDP
protocol by default (although this is configurable, see
Section~\ref{sec:Configuration} for details), as the cost of sending UDP packets
is very low cost when compared to TCP, and we do not require the deliverability
guarantees of TCP (metrics are non-critical). Metrics are typically sent over a
\gls{lan}, where packet loss is usually quite low. In the event that is it not,
but the packet loss is uniform, the loss of some metrics should not affect the
overall data trend.

\begin{listing}
  \centering
  \mintinline{text}{gorets:1|c\nglork:320|ms\ngaugor:333|g\nuniques:765|s}
  \caption{Multiple metrics in a single packet}
  \label{lst:statsdMultiplePackets}
\end{listing}

\subsection{InfluxDB}
\label{sub:InfluxDB}

InfluxDB is an open-source, NoSQL database, optimised for the storage and
retrieval of \gls{time-series data}. Influx is part of a family of databases
designed to store (amongst other things) a large amount of metrics data at very
high speed (the order of millions of events per second), which also includes
\href{http://graphite.wikidot.com/}{Graphite} and
\href{http://opentsdb.net/}{OpenTSDB}. InfluxDB was chosen for this project due
to its maturity (compared to OpenTSDB) and performance (compared to
Graphite/Carbon).

\subsection{Grafana}
\label{sub:Grafana}

\href{http://grafana.org/}{Grafana} is an open-source visualisation tool for
\gls{time-series data}, typically that which is produced by Internet
infrastructure and application analytics. Developed by a consortium of Internet
companies as a modern replacement for the previously popular
\href{https://github.com/graphite-project/graphite-web}{Graphite-Web} project.
Grafana can visualise \gls{time-series data} from a variety of data-sources,
such as the ElasticSearch, the aforementioned Graphite and, most importantly,
InfluxDB.

\missingfigure[figcolor=white]{Grafana screenshot}

\subsection{Docker}
\label{sub:Docker}

Gamq is designed to send StatsD messages (Section~\ref{sub:StatsD}) across the
network. In a production environment, these would be sent to the above software
stack, running on a separate server in the datacenter. In a development
environment, the metrics stack could be run on a development server, or the
developers local machine. The configured stack should therefore be as portable
as possible, to make development as simple as possible. \todo{Docker description}

\section{Configuration}
\label{sec:Configuration}

As with most pieces of software - there are situations where the default
behaviour of a message broker requires adjustment to suit its operating
environment. One example of this is the network port that the broker operates
on, and which other applications will connect to in order to exchange messages.
Whilst the default port (48879\footnote{A number chosen due to its memorable
hexadecimal representation: \texttt{0xBEEF}} in the case of this broker) may be
suitable in most cases, since no other 'well-known' applications use that
particular
port.\footnote{\url{https://en.wikipedia.org/wiki/List_of_TCP_and_UDP_port_numbers}}.

\subsection{Command-line configuration}
\label{sub:Command-line configuration}

A number of these configurable parameters make sense to expose as command-line
parameters, specified at application start time. The available parameters can be
exposed using the \texttt{--help} parameter, the output of which can be seen in
Listing~\ref{lst:gamqHelpOutput}

\begin{listing}[ht]
  \centering
  \inputminted{bash}{code/gamqHelpOutput}
  \caption{Output of running the broker with the --help flag}
  \label{lst:gamqHelpOutput}
\end{listing}

\subsection{File-based configuration}
\label{sub:File-based configuration}

Whilst specifying arguments on the command line gives flexibility, there are
certain options that, whilst configurable, are either too numerous, or change
too infrequently to justify command-line flags. One major example of this is the
log configuration for the broker - specifying the format, and location of
messages logged using the \href{https://github.com/cihub/seelog}{Seelog}
library. An example log configuration can be seen in
Listing~\ref{lst:seelogConfig}.

\begin{listing}[ht]
  \centering
  \inputminted{xml}{code/gamq/config/logconfig.xml}
  \caption{Example Seelog configuration file for gamq.}
  \label{lst:seelogConfig}
\end{listing}

\section{Code Structure}
\label{sec:codestructure}

\todo[inline]{Box-level overview of how the code is structured/how messages
flow through the broker}

\chapter{Design}
\label{chap:Design}

\section{Project philosophy}
\label{sec:Project philosophy}

One of the learning objectives for this project, was to discover and utilise
\gls{foss} technologies wherever possible, as well as give back to the
\gls{foss} community. As such, all of the tooling used in the production of the
project, as well as the project itself were open-source. The project itself is
published under the MIT License, which allows anyone to "use, copy, modify,
merge, publish, distribute, sublicense, and/or sell copies of the Software",
without restriction.

\section{Presentation interface}
\label{sec:presentation}

As a (primarily) middleware project - coming up with a compelling presentation
interface is crucial to allowing people to engage with the software, as well as
providing the ability to monitor the broker, should it be deployed in a
production environment. Broker metrics such as:

\begin{itemize}
  \item Messages/second (broken down by topic)
  \item (Average) End-to-end latency for each message
  \item Memory/Disk usage
  \item CPU utilisation
  \item Pending messages
  \item Total messages processed
\end{itemize}

\noindent
And client metrics such as:

\begin{itemize}
  \item Messages sent
  \item Messages received
  \item Messages lost
\end{itemize}

Must be easily accessible, and visualised. Details on how this was implemented
are given in Section~\ref{sec:presentationInterface}.

\section{Configuration}
\label{sec:Configuration}

As with most pieces of software - there are situations where the default
behaviour of a message broker requires adjustment to suit its operating
environment. One example of this is the network port that the broker operates
on, and which other applications will connect to in order to exchange messages.
Whilst the default port (48879\footnote{A number chosen due to its memorable
hexadecimal representation: \texttt{0xBEEF}} in the case of this broker) may be
suitable in most cases, since no other 'well-known' applications use that
particular
port\footnote{\url{https://en.wikipedia.org/wiki/List_of_TCP_and_UDP_port_numbers}} -
there may be certain environments where it is necessary to change it. This can
be acheived by editing the relevant gamq configuration.

\subsection{Command-line configuration}
\label{sub:Command-line configuration}

A number of these configurable parameters make sense to expose as command-line
parameters, specified at application start time. The available parameters can be
exposed using the \texttt{--help} parameter, the output of which can be seen in
Listing~\ref{lst:gamqHelpOutput}

\begin{listing}[ht]
  \centering
  \RecustomVerbatimEnvironment{Verbatim}{BVerbatim}{}
  \inputminted{bash}{code/gamqHelpOutput}
  \caption{Output of running the broker with the --help flag}
  \label{lst:gamqHelpOutput}
\end{listing}

\subsection{File-based configuration}
\label{sub:File-based configuration}

Whilst specifying arguments on the command line gives flexibility, there are
certain options that, whilst configurable, are either too numerous, or change
too infrequently to justify command-line flags. One major example of this is the
log configuration for the broker - an \gls{xml} file specifying specifying the
format, and location of messages logged using the
\href{https://github.com/cihub/seelog}{Seelog} library. An example log
configuration can be seen in Listing~\ref{lst:seelogConfig}, which defines the
format and location of log messages written by gamq.

\begin{listing}[ht]
  \centering
  \RecustomVerbatimEnvironment{Verbatim}{BVerbatim}{}
  \inputminted{xml}{code/gamq/config/logconfig.xml}
  \caption{Example Seelog configuration file for gamq.}
  \label{lst:seelogConfig}
\end{listing}

\section{Code Structure}
\label{sec:codestructure}

\begin{figure}[H]
  \centering
  \begin{tikzpicture}[thick, scale=0.6, every node/.style={scale=0.6}]
  \node(connectionManager) [draw,rectangle,minimum width=1cm,minimum height=1cm]{Connection Manager};

  \node(queueManager) [draw,rectangle,minimum width=1cm,minimum height=1cm, right=1cm of connectionManager]{Queue Manager};

  \node(messageMutator1) [draw,rectangle,gray,minimum height=1cm,above right=1cm and 1cm of queueManager]{(0...n) Message Mutator};
  \node(messageMutator2) [draw,rectangle,gray,minimum height=1cm,right=1cm of queueManager]{(0...n) Message Mutator};
  \node(messageMutator3) [draw,rectangle,gray,minimum height=1cm,below right=1cm and 1cm of queueManager]{(0...n) Message Mutator};

  \node(client1) [draw,rectangle,red,minimum width=1cm,minimum height=1cm, above left=1cm and 1cm of connectionManager]{Client 1};
  \node(client2) [draw,rectangle,red,minimum width=1cm,minimum height=1cm, left=1cm of connectionManager]{Client 2};
  \node(client3) [draw,rectangle,red,minimum width=1cm,minimum height=1cm, below left=1cm and 1cm of connectionManager]{Client 3};

  \node(queue1) [draw,rectangle,minimum width=1cm,minimum height=1cm, right=1cm of messageMutator1]{Queue 1};
  \node(topic1) [draw,rectangle,minimum width=1cm,minimum height=1cm, right=1cm of messageMutator2]{Topic 1};
  \node(queue2) [draw,rectangle,minimum width=1cm,minimum height=1cm, right=1cm of messageMutator3]{Queue 2};

  \node(shipper1) [draw,rectangle,minimum width=1cm,minimum height=1cm, right=1cm of queue1]{Message Shipper};
  \node(shipper2) [draw,rectangle,minimum width=1cm,minimum height=1cm, right=1cm of topic1]{Message Shipper};
  \node(shipper3) [draw,rectangle,minimum width=1cm,minimum height=1cm, right=1cm of queue2]{Message Shipper};

  \node(metricManager) [draw,rectangle,minimum width=1cm,minimum height=1cm, above=2cm of messageMutator1]{Metric Manager};

  \draw[->] (client1) edge (connectionManager);
  \draw[->] (client2) edge (connectionManager);
  \draw[->] (client3) edge (connectionManager);

  \draw[->] (connectionManager) edge (queueManager);
  \draw[->] (queueManager) edge (messageMutator1);
  \draw[->] (queueManager) edge (messageMutator2);
  \draw[->] (queueManager) edge (messageMutator3);
  \draw[->] (messageMutator1) edge (queue1);
  \draw[->] (messageMutator2) edge (topic1);
  \draw[->] (messageMutator3) edge (queue2);

  \draw[->] (queue1) edge (shipper1);
  \draw[->] (topic1) edge (shipper2);
  \draw[->] (queue2) edge (shipper3);

  % Metrics Manager connections
  \draw[->] (connectionManager) edge[bend left, gray] (metricManager);
  \draw[->] (queueManager) edge[bend left, gray] (metricManager);
  \draw[->] (messageMutator1) edge[gray] (metricManager);
  \draw[->] (queue1) edge[bend right, gray] (metricManager);
  \draw[->] (shipper1) edge[bend right, gray] (metricManager);

\end{tikzpicture}

  \caption{A block diagram of the system components, and their interactions}
  \label{fig:systemBlockDiagram}
\end{figure}

An overview of the project structure can be seen in
Figure~\ref{fig:systemBlockDiagram}. The communications between each system
block (shown in black) are handled via GoChannels
(Section~\ref{sub:golangConcurrency}). The responsibilities of each block are as
follows:

\begin{description}[style=nextline]
  \item[Connection Manager]
  Handle incoming and open connections for each of the broker protocols, and
  maintain details of connected clients. The protocol used for client
  connections is abstracted inside ConnectionManager, so that a client can
  transparently use \gls{tcp}, \gls{udp} - or some other protocol (such as
  \gls{http}, or \gls{amqp}). Connection Manager parses received commands
  (Section~\ref{sec:protocol}), and calls the relevant methods on QueueManager.
  \item[Queue Manager]
  Maintains a list of active queues/topics, and handles the publishing of
  messages to said queues/topics, as well as the addition/removal of
  subscribers.
  \item[Message Mutators]
  An optional module for each queue/topic, which allows simple mutations to be
  performed on messages as they pass through the broker. One example could be
  throwing away values below/above a certain threshold, or enriching data as it
  passes through the data, with information from an external database. Multiple
  message mutators can be chained together, should multiple mutations be
  required.
  \item[Queue/Topics]
  The buffering datastructures designed to hold and deliver messages to
  consumers. Due to the semantic differences between queues and topics
  (Sections~\ref{sub:queues} and \ref{sub:topics}), subtly different
  datastructures are required for each - details of which are given in
  Section~\ref{sub:messageStorage}.
  \item[Message Shippers]
  Responsible for managing the delivery of messages to subscribers.
  \item[Metric Manager]
  Responsible for collecting metrics about broker performance from each of the
  broker components (latency, throughout, number of connected clients etc.), and
  asynchronously ship them to \ref{sub:StatsD} for visualisation and analysis.
\end{description}

The implementation details for each of these components are explored in
(Section~\ref{chap:Implementation}).

\section{Protocol Design}
\label{sec:protocol}

As a piece of network-connected software - a clear, well-defined protocol is
essential to allow other networked applications to interact with the messaging
broker. Although a simple protocol is used in the current implementation, the
message decoding logic is designed to be somewhat modular, allowing for
additional protocols to easily be supported. The NATS message
protocol\cite{natsProtocol} served as an inspiration for gamq - as it showed it
was possible for text-based protocols to be as performant as customised binary
protocols, and had a slew of other benefits, such as human-readability, and ease
of debugging.

The gamq protocol defines several commands, which can be seen in
Table~\ref{tab:protocol}. These commands are all delimited by \verb|\r\n|, which
is the telnet standard. The two most frequently used commands (\emph{PUB} and
\emph{SUB}) are intentionally short to reduce the overhead required to send them
over the wire - which is especially important as message rate increases. Note
that whilst all commands listed in the following sections are demonstrated using
telnet, in reality they would be given programatically, by software wishing to
send messages to the broker (An example of which can be seen in
\ref{chap:benchmarkCode})

\begin{table}[H]
\centering
\caption{The gamq protocol commands}
\label{tab:protocol}
\begin{tabular}{|l|}
\hline
Command \\ \hline
\verb|HELP\r\n|                           \\
\verb|PUB <queue>\r\n <message>\r\n.\r\n| \\
\verb|SUB <queue>\r\n|                    \\
\verb|PINGREQ\r\n|                        \\
\verb|DISCONNECT\r\n|                     \\
\verb|SETACK <on/off>\r\n|                \\ \hline
\end{tabular}
\end{table}

\subsection{HELP}
\label{sub:helpCommand}

\begin{listing}[H]
  \centering
  \RecustomVerbatimEnvironment{Verbatim}{BVerbatim}{}
  \inputminted{bash}{code/gamqHelpOutputTelnet}
  \caption{Output from the gamq HELP command}
  \label{lst:gamqHelpTelnet}
\end{listing}

Print the help message (as seen in Listing~\ref{lst:gamqHelpTelnet})

\subsection{PUB}
\label{sub:pubCommand}

\begin{listing}[H]
  \centering
  \RecustomVerbatimEnvironment{Verbatim}{BVerbatim}{}
  \inputminted{bash}{code/gamqPubOutput}
  \caption{Publishing a message to gamq}
  \label{lst:gamqPubTelnet}
\end{listing}

Send the given message, to the given queue. In Listing~\ref{lst:gamqPubTelnet},
we are publishing a message to the queue 'abc'. Note that messages can span
multiple lines, and are terminated with a single '.', on a line by itself.

\subsection{SUB}
\label{sub:subCommand}

\begin{listing}[H]
  \centering
  \RecustomVerbatimEnvironment{Verbatim}{BVerbatim}{}
  \inputminted{bash}{code/gamqSubOutput}
  \caption{Subscribing to a queue/topic in gamq}
  \label{lst:gamqSubTelnet}
\end{listing}

Subscribe to messages posted on the named queue (see Section~\ref{sub:pubsub}).
Note that in Listing~\ref{lst:gamqSubTelnet}, when we subscribe to the queue
'abc', we receive the message sent to that queue in
Section~\ref{sub:pubCommand}.

\subsection{PINGREQ}
\label{sub:pingreqCommand}

\begin{listing}[H]
  \centering
  \RecustomVerbatimEnvironment{Verbatim}{BVerbatim}{}
  \inputminted{bash}{code/gamqPingreqOutput}
  \caption{Pinging the broker through telnet}
  \label{lst:gamqPingreqTelnet}
\end{listing}

Ask the broker to return a 'PINGRESP' message. This can be used to check that
the connection to the broker is active, and the broker is responding to
messages. Clients should send this at regular intervals.

\subsection{DISCONNECT}
\label{sub:disconnectCommand}

\begin{listing}[H]
  \centering
  \RecustomVerbatimEnvironment{Verbatim}{BVerbatim}{}
  \inputminted{bash}{code/gamqDisconnectOutput}
  \caption{Closing a connection to the broker}
  \label{lst:gamqDisconnectTelnet}
\end{listing}

Unsubscribe from all subscribed queues, and close the connection to the broker.
Whilst the client can achieve the same effect by simply closing its side of a
TCP connection\todo{\gls{tcp}/\gls{udp} section}, since \gls{udp} is a connectionless protocol, it
requires an explicit disconnect message (as in
Listing~\ref{lst:gamqDisconnectTelnet})

\subsection{SETACK}
\label{sub:setackCommand}

\begin{listing}[H]
  \centering
  \RecustomVerbatimEnvironment{Verbatim}{BVerbatim}{}
  \inputminted{bash}{code/gamqSetackOutput}
  \caption{Demonstrating the effect of the SUBACK command}
  \label{lst:gamqSubackTelnet}
\end{listing}

Turn the explicit acknowledgement of messages on/off (As demonstrated in
Listing~\ref{lst:gamqSubackTelnet}). Clients can use the acknowledgement to
detect potentially lost messages using a timeout, and optionally retransmit the
lost message if at-least-once semantics (Section~\ref{sub:failureHandling}) are
required. Explicitly acknowledging messages (as opposed to relying on built in
\gls{tcp} error correction) can have a significant overhead, but can also
dramatically reduce the number of lost messages when using a protocol with no
built-in error correction(\gls{udp}).
